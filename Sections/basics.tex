\chapter{Basic properties of Integers}

\section{Divisibility and primality}

\bt{``$a$ divides $b$''}, i.e., $\left(\frac{b}{a}\right)$, means \bt{$b$} is reached by \bt{$a$}, when \bt{$a$} is multiplied by some integer.

\begin{Def}[Division]

    Let $a,b,x\in\mathbb{Z}$: $\left(\frac{b}{a}\right)$ means ``$b=ax$''.\\

    \noindent
    \underline{\textbf{Denoted:} $a|b$,}\\
    $\qquad$ read ``$a$ divides $b$,'' and ``$a$ doesn't divide $b$'' is, $a\nmid b$.\\
    \noindent
\end{Def}

\noindent
\textbf{Examples:}
\begin{itemize}
    \item $3\mid 6$ because $6=3\cdot2$.
    \item $3\nmid5$ because $5\neq3\cdot x$ for any $x\in\mathbb{Z}$.
    \item $2\mid 0$ because $0=2\cdot0$.
    \item $0\nmid2$ because $2\neq0\cdot x$ for any $x\in\mathbb{Z}$.
\end{itemize}


\begin{Note}
    \textbf{Note:} $a,b,x\in\mathbb{Z}$ for, ``$\left(\frac{b}{a}\right)$'' or ``$b = ax$'' are labeled, $a$: \bt{divisor}, $b$: \bt{dividend}, $x$: \bt{quotient}.
\end{Note}

\begin{Tip}
    Many problems will involve manipulating this ``$b=ax$'' equation. Whether
    it's substituting $b$ for $ax$ or vice-versa, or adding/subtracting/multiplying/dividing ``$b=ax$''
    to itself to reveal some property.\\

    \noindent
    Many definitions and theorems will relate to each other or build off one another.
    It's crucial to understand what concepts mean rather than memorizing them. This means
    the ability to derive theorems or definitions from scratch, based on intuitive
    understanding of the content.
\end{Tip}

\newpage

\noindent
Observe the following:
\begin{theo}[Properties of divisibility]

    \textbf{Theorem 1.1.} For all $a, b, c \in \mathbb{Z}$:

    \begin{itemize}
        \item[(i)] $a \mid a$, $1 \mid a$, and $a \mid 0$;
        \item[(ii)] $0 \mid a$ if and only if $a = 0$;
        \item[(iii)] $a \mid b$ if and only if $-a \mid b$ if and only if $a \mid -b$;
        \item[(iv)] $a \mid b$ and $a \mid c$ implies $a \mid (b + c)$;
        \item[(v)] $a \mid b$ and $b \mid c$ implies $a \mid c$.
    \end{itemize}
\end{theo}

To prove,

\begin{Proof}[Theorem 1.1]
    \begin{itemize}
        \item hi
    \end{itemize}
\end{Proof}