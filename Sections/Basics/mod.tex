\section{Modular Arithmetic \& Residues}
\noindent
\textbf{Remember:} For $a\in\mathbb{R}$, \underline {$a\in[0,1)$ is a range}, i.e., including decimals from 0 to 1 (excluding 1).\\
\begin{Def}[Floor \& Ceiling]

    \label{def:floor_ceiling}

    For $x\in\mathbb{R}$, $m,n\in\mathbb{Z}$ we map $\mathbb{R}\rightarrow\mathbb{Z}$,\\

    \noindent
    \textbf{Floor} $x$, $\floor{x}$, is the largest $m$ such that $m\leq x<m+\varepsilon$, where $\varepsilon\in[0,1)$.\\

    \vspace{-.5em}
    \noindent
    \textbf{Ceiling} $x$, $\ceil{x}$, is the smallest $n$ such that $n-\varepsilon<x\leq n$, where $\varepsilon\in[0,1)$.\\


\end{Def}
\noindent

\begin{Def}[Mod Operator]

    \label{def:mod_operator}

    Let $a,b\in\mathbb{Z}$, $b\>0$: The remainder of $a$ divided by $b$. I.e., $a-b\floor{\frac{a}{b}}$.\\

    \noindent
    \textbf{Denoted:} ``$a\bmod b$'' or ``$a\,\%\,b$''.
\end{Def}

\noindent
\textbf{Examples:} $8\bmod3=2$, and $5\bmod2=1$\\

\noindent
From The Division Algorithm (\ref{theo:division_algorithm}) we see for all $a,b\in\mathbb{Z}$, $b\neq0$:
$a=bq+r$ then $r=a\bmod b$.

\begin{Proof}[Mod Operator]
    \underline{The Division Algorithm (\ref{theo:division_algorithm}) only works for $b>0$.} To generalize for $b<0$,
    \begin{align*}
        a=bq+r \quad \quad    & \textit{ Given}                    \\
        a/b=q+r/b \quad \quad & \textit{ Divide both sides by $b$}
    \end{align*}
    \noindent
    We know $0\leq r<b$, dividing $b$ yielded $0\leq \dfrac{r}{b}<1$, so
    $$\dfrac{r}{b}\in[0,1)\in\mathbb{R}$$
    We notice $q=\left\lfloor\dfrac{a}{b}\right\rfloor$, as $q$ is the largest integer that fits into $a$, $b$ times.
\end{Proof}
\begin{Tip}
    $q=\left\lfloor\dfrac{a}{b}\right\rfloor$ is similar to integer division in programming languages.
\end{Tip}

\newpage

\begin{theo}[Division Algorithm Extended]

    Let $a, b \in \mathbb{Z}$ with $b > 0$, and let $x \in \mathbb{R}$. Then there exist unique $q, r \in \mathbb{Z}$ such that $a = bq + r$ and $r \in [x, x + b)$.
\end{theo}

\noindent
$r \in [x, x + b)$ allows us to work with negative numbers and different intervals. Let's
try to build some intuition about division and remainders.

\begin{center}
    $S=\{a-bx:q\in\mathbb{Z}\}$, $a=6$, $b=2$, remainder output:
\end{center}
\begin{center}
    \begin{tabular}{c|cc}
        $x$ & $a-bx$                \\
        \hline
        0   & 0      & $=6-2\cdot0$ \\
        1   & 4      & $=6-2\cdot1$ \\
        2   & 2      & $=6-2\cdot2$ \\
        3   & 0      & $=6-2\cdot3$ \\
        \hline
        4   & -2     & $=6-2\cdot4$ \\
        5   & -4     & $=6-2\cdot5$ \\
        6   & -6     & $=6-2\cdot6$ \\
        7   & -8     & $=6-2\cdot7$ \\
    \end{tabular}
\end{center}

\noindent
Attempt to divide two numbers varying the divisor.
You'll notice the remainder will always be between 0 and the divisor (exclusive).



\noindent
\begin{minipage}{0.32\textwidth}
    \centering
    \begin{tabular}{c|c}
        $b$ & $3 \bmod b$ \\
        \hline
        1   & 0           \\
        2   & 1           \\
        3   & 0           \\
        4   & 3           \\
        5   & 3           \\
        6   & 3           \\
        7   & 3           \\
        8   & 3           \\
    \end{tabular}
\end{minipage}%
\begin{minipage}{0.32\textwidth}
    \centering
    \begin{tabular}{c|c}
        $b$ & $9 \bmod b$ \\
        \hline
        1   & 0           \\
        2   & 1           \\
        3   & 0           \\
        4   & 1           \\
        5   & 4           \\
        6   & 3           \\
        7   & 2           \\
        8   & 1           \\
        9   & 0           \\
        10  & 9           \\
    \end{tabular}
\end{minipage}%
\begin{minipage}{0.32\textwidth}
    \centering
    \begin{tabular}{c|c}
        $b$ & $7 \bmod b$ \\
        \hline
        1   & 0           \\
        2   & 1           \\
        3   & 1           \\
        4   & 3           \\
        5   & 2           \\
        6   & 1           \\
        7   & 0           \\
        9   & 7           \\
    \end{tabular}
\end{minipage}

\vfill
Do you see any patterns?
\newpage

Let's group the remainders by the divisor, vise versa.\\


\noindent
\begin{minipage}{0.32\textwidth}
    \centering
    \begin{tabular}{c|c}
        $r$ & $3 \bmod b$   \\
        \hline
        0   & 1, 3          \\
        1   & 2             \\
        3   & 4, 5, 6, 7, 8 \\
    \end{tabular}
\end{minipage}%
\begin{minipage}{0.32\textwidth}
    \centering
    \begin{tabular}{c|c}
        $r$ & $9 \bmod b$ \\
        \hline
        0   & 1, 3, 9     \\
        1   & 2, 4, 8     \\
        3   & 5, 6, 7     \\
        9   & 10          \\
    \end{tabular}
\end{minipage}%
\begin{minipage}{0.32\textwidth}
    \centering
    \begin{tabular}{c|c}
        $r$ & $7 \bmod b$ \\
        \hline
        0   & 1, 7        \\
        1   & 2, 6        \\
        3   & 4           \\
        2   & 5           \\
        7   & 9           \\
    \end{tabular}
\end{minipage}
