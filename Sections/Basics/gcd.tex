\newpage
\section{Primes \& Greatest/Lowest Common Divisors}

\begin{Def}[Greatest Common Divisor (GCD)]

    For all $a,b\in\mathbb{Z}$,\\
    The \textit{greatest common divisor} of $a$ and $b$, is the largest positive integer dividing both $a$ and $b$.\\
    I.e., $d\in\mathbb{Z}: d\mid a$ and $d\mid b$, and \underline{$d$ is unique.}

    \noindent
    Denoted: $\gcd(a,b)$.
\end{Def}

\begin{Proof}[GCD Existence and Uniqueness]

    \label{theo:gcd_existence_uniqueness}

    Let $a,b\in\mathbb{Z}$, and $d=\gcd(a,b)$.\\
    \begin{itemize}
        \item  \textbf{Existence:} $d$ exists by the Well-Ordering Principle, 
        as it's greatest element in the set of common divisors of $a$ and $b$.
        \item \textbf{Uniqueness:} Let there be another GCD $d'\in\mathbb{Z}$ such that $d'\mid a$ and $d'\mid b$.\\
        Then, $d'\mid d$ and $d\mid d'$, so $d=\pm d'$ (\ref{theo:properties_of_divisibility}). GCD must be positive, so $d=d'$.
    \end{itemize}

   

\end{Proof}

\begin{theo}[GCD Ideal Linear Combination of $\mathbb{Z}$]

    \label{theo:gcd_ideal_linear_combination}

    For all \(a, b,d \in \mathbb{Z}\) and $d=\gcd(a,b)$: \(a\mathbb{Z} + b\mathbb{Z} = d\mathbb{Z}\)

\end{theo}

\begin{Proof}[GCD Ideal Linear Combination of $\mathbb{Z}$]

    \label{proof:gcd_ideal_linear_combination}

    Let $I:=a\mathbb{Z} + b\mathbb{Z}$. Then there exists $c\in Z$ such that $c\mathbb{Z} = I$ (\ref{theo:ideal_generator}). Then $a,b,c\in I$, are all
    positive integers. We will prove facts of $c$:
    \begin{itemize}
        \item \textbf{Common Divisor: }$a,b\in I$ and $c\mathbb{Z}=I$. So $a,b\in c\mathbb{Z}$. Then $c\mid a$ and $c\mid b$ (\ref{theo:ideal_properties}).
        \item \textbf{Linear Combination:} Since $c\in I$ and $a\mathbb{Z} + b\mathbb{Z} = I$. There exists some linear combination $\underline{as+bt=c}$ for some $s,t\in\mathbb{Z}$ (\ref{def:ideal_operations}).
        \item \textbf{Greatest Divisor} Let $a,b\in I$ be the products $a=a'c'$ and $b=b'c'$, where $a',b'\in\mathbb{Z}$.
              Then there's a linear combination $a'c'+b'c'=c'(a'+b')=c$. So $c\mid c'$, hence $c$ is the greatest common divisor of $a$ and $b$.
        \item \textbf{Uniqueness:} By Lemma (\ref{theo:gcd_existence_uniqueness}), $c$ is unique.
    \end{itemize}
\end{Proof}

\newpage

\noindent
This next theorem heavily relies on Definition (\ref{theo:ideal_properties}) and the previous Proof (\ref{theo:gcd_ideal_linear_combination}).
\begin{theo}[Element Linear Combinations of $\mathbb{Z}$]

    For all \(a, b,d \in \mathbb{Z}\) and $d=\gcd(a,b)$:\\
    There exists some \(s, t \in \mathbb{Z}\), such that \(as + bt = r\) if and only if $d\mid r$.
        
\end{theo}


\begin{Proof}[Element Linear Combinations of $\mathbb{Z}$]

    Let $r\in\mathbb{Z}$, and $d=\gcd(a,b)$, we have
    
    \begin{center}

        \begin{tabular}{p{2cm} p{1cm} p{2cm} p{5.5cm}}
            $as + bt = r$ & $\Longleftrightarrow$ & $r \in a\mathbb{Z} + b\mathbb{Z}$ & (Ideal Multiplicative Closure (\ref{def:ideal_operations})) \\
            & $\Longleftrightarrow$ & $r \in d\mathbb{Z}$ & (GCD Linear Combination (\ref{theo:gcd_ideal_linear_combination})) \\
            & $\Longleftrightarrow$ & $d \mid r$ & (Property of Ideals (\ref{theo:ideal_properties})) \\
        \end{tabular}
    \end{center}
\end{Proof}

\noindent
\begin{Note}
    \textbf{Note:} In $as + bt = r$, $s$ and $t$ are not unique, \underline{nor do they have to be positive}: Example (\ref{def:ideal_operations})
\end{Note}
