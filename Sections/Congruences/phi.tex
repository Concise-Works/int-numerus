\subsection{Euler's Phi Function}
\begin{Tip}
    \textbf{Leonhard Euler} (1707–1783), pronounced as ``oiler,'' was a Swiss mathematician born in Basel. He worked in St. Petersburg and Berlin, shaping calculus and number theory.
\end{Tip}

\noindent
Also known as the \textbf{Euler Totient Function}:
\begin{Def}[Euler's Phi Function]

    For all $n\in\mathbb{Z^+}$, we define Euler's Phi Function as:
    \[\varphi (n):=|\mathbb{Z}_n^*|\]
    \noindent
    The number of inverses modulo $n$. Numbers coprime to $n$ are in $\mathbb{Z}_n^*$.\\
     Therefore, for primes $p$, $\varphi(p)=p-1$.
\end{Def}

\begin{theo}[Chinese Remainder's Phi Function]

    \label{theo:chinese_remainder_phi}

    Let $n:=\prod_{i=1}^k n_i$ be the product of pairwise coprime integers. Then:
    \[ \varphi(n) = \prod_{i=1}^k \varphi(n_i) = \varphi(n_1)\cdot\varphi(n_2)\cdots\varphi(n_k)\]
    \noindent
    The number of inverses in $\mathbb{Z}_n^*$ is the product of the number of inverses in $\mathbb{Z}_{n_i}^*$.
\end{theo}
\begin{Proof}[Chinese Remainder's Phi Function]
    Consider the Chinese Remainder Map $\theta:\mathbb{Z}_n\to\mathbb{Z}_{n_1}\times\mathbb{Z}_{n_2}\times\cdots\times\mathbb{Z}_{n_k}$.
    Since $\theta$ is isomorphic, it has a one-to-one correspondence. If we restrict our input to $\mathbb{Z}_n^*$, then the output will be in $\mathbb{Z}_{n_1}^*\times\mathbb{Z}_{n_2}^*\times\cdots\times\mathbb{Z}_{n_k}^*$.
    Hence, $|\mathbb{Z}_n^*|=|\mathbb{Z}_{n_1}^*|\times|\mathbb{Z}_{n_2}^*|\times\cdots\times|\mathbb{Z}_{n_k}^*|=\prod_{i=1}^k|\mathbb{Z}_{n_i}^*|=\prod_{i=1}^k\varphi(n_i)$.

\end{Proof}
\begin{theo}[Euler's Phi of a Raised Prime]

    Let $p$ be a prime and $e\in\mathbb{Z^+}$. Then:
    \[\varphi(p^e)=p^{e-1}(p-1)\]
\end{theo}
\begin{Proof}[Euler's Phi of a Raised Prime]
    
    $\varphi(n)$ counts residue classes in $\mathbb{Z}_n$ that are coprime to $n$. $\mathbb{Z}_n$ represent integers $[0, n-1]$.\\

    \noindent
    Examining $\mathbb{Z}_{p^e}$, to obtain coprimes, we omit members sharing common factors to $p^e$, i.e.,\\
    multiples $p$, which $p^e$ gives us $e$ of.\\

    \noindent
    Since the last factor reaches $p^e$, we ignore it, as it's beyond $p^e-1$. Leaving us $p^{e-1}$ multiples.\\
    Therefore, $\varphi(p^e)=p^e-p^{e-1}=p^{e-1}(p-1)$.
    
\end{Proof}

\noindent
As implied by Theorem \ref{theo:chinese_remainder_phi}, we can generalize this to the prime factorization of $n$.
\begin{theo}[Phi of Prime Factorization]

    Let $n:=\prod_{i=1}^k p_i^{e_i}$ be the prime factorization of $n$. $\{p_{i}^{e_i}\}$ are pairwise coprime. Then:
    \[\varphi(n)=\prod_{i=1}^k p_i^{e_i-1}(p_i-1)\]
    Expanding the product, 
    \[
\varphi(n) = p_1^{e_1} \cdot \left( 1 - \frac{1}{p_1} \right) \cdot p_2^{e_2} \cdot \left( 1 - \frac{1}{p_2} \right) \cdot \dots \cdot p_k^{e_k} \cdot \left( 1 - \frac{1}{p_k} \right)
\]
\noindent
Which gives us:
    \[\varphi(n)=n\prod_{i=1}^k\left(1-\frac{1}{p_i}\right)\]
\noindent
as $n$ represents $p_1^{e_1}p_2^{e_2}\cdots p_k^{e_k}$.
\end{theo}
    
    