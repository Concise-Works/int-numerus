\subsection{Chinese Remainder Map Applied to Quadratic Residues}
Revisiting the Chinese Remainder Map (\ref{theo:crt_map}), we show that since 
prime factorizations also follow the Chinese Remainder Theorem schema (a product of pairwise coprime elements),
we can generate a congruence system. Also allowing us to extend such systems to residue classes.
\vfill
\begin{center}
    \textit{Continued on next page...}
\end{center}
\vfill
\begin{Tip}
    The Chinese Remainder Theorem (CRT) is named after its origin in ancient China, where it first appeared in the work of the mathematician \textbf{Sunzi} in the 3rd century AD.
    In his book \textit{Sunzi Suanjing}, he posed a problem involving finding a number that leaves specific remainders when divided by different moduli. Although the method was later formalized in modern mathematics, the name honors its roots in early Chinese mathematical texts.
\end{Tip}

\newpage
\begin{theo}[Chinese Remainder Map of Prime Factorizations]

    Let $n\in\mathbb{Z}^+:2\nmid n$ and $n>1$ be a product of pairwise coprime elements,
    \[n = p_1^{e_1}p_2^{e_2}\cdots p_k^{e_k},\]
    We map our factorization via CRT $\theta:=\Z_n\to\Z_{p_1^{e_1}}\times\cdots\times\Z_{p_k^{e_k}}$. Then 
    we take $\alpha\in\Z_n^*$ as, $\alpha:=[a]_n$ and $\gcd(a,n)=1$, allowing us to construct the map $\theta(\alpha)=\{\alpha_1,...,\alpha_k\}$:
    \begin{align*}
        \theta(\alpha) = \begin{cases}
           \alpha_1 & \text{mod } p_1^{e_1} \\
            \alpha_2 & \text{mod } p_2^{e_2} \\
            \vdots & \vdots \\
            \alpha_k & \text{mod } p_k^{e_k}
        \end{cases}
    \end{align*}
    \noindent
    Which is the system of congruences:
    \begin{align*}
        \alpha &\equiv \alpha_1 \pmod{p_1^{e_1}} \\
        \alpha &\equiv \alpha_2 \pmod{p_2^{e_2}} \\
        &\hspace*{.5em}\vdots \\
        \alpha &\equiv \alpha_k \pmod{p_k^{e_k}}
    \end{align*}
    \noindent
    Meaning if we have some $\beta\in\Z_n^*: \alpha=\beta^2$, if $\theta(\beta)=\{\beta_1,...,\beta_k\}$, then:
    \[
        (\alpha_1,...,\alpha_k) = \theta(\alpha) = \theta(\beta^2) = \theta(\beta)^2 = (\beta_1^2,...,\beta_k^2)
    \]
    \noindent
    Then $\alpha_i = \beta_i^2$ for each $i=1,...,k$. Suppose we began $\alpha_i=\beta_i^2$, for some $\beta_i\in\mathbb{Z}_{p_i^{e_i}}$.
    Then $(\beta_1,\dots,\beta_k)$ is $\theta(\beta)^{-1}$. Then,
    \[ 
        (\beta_1,...,\beta_k)^2 = \theta(\beta^2) = (\beta_1^2,...,\beta_k^2) = (\alpha_1,...,\alpha_k) = \theta(\alpha)
    \]
    \noindent
    Revealing for each $i=1,...,k$: \Large
     \[\alpha\in(\mathbb{Z}_n^*)^2\Longleftrightarrow\alpha_i\in(\mathbb{Z}_{p_i^{e_i}}^*)^2\]
    \normalsize
    Restricting $\theta$ to $(\mathbb{Z}_n^*)^2$, we get $\theta:(\mathbb{Z}_n^*)^2\to(\mathbb{Z}_{p_1^{e_1}}^*)^2\times\cdots\times(\mathbb{Z}_{p_k^{e_k}}^*)^2$. Then if $\alpha\in\mathbb{Z}_n^*$, there is some 
    tuple $\{\alpha_1,...,\alpha_k\}$ that it maps to, and vise versa, showing a bijection. Leaving us with:
    \LARGE
    \[
    |(\mathbb{Z}_n^*)^2| = \prod_{i=1}^{k} \left(\varphi(p_i^{e_i})/2 \right) = \varphi(n)/2^k
    \]
    \normalsize
    By theorems (\ref{theo:cardinality_peres}) and (\ref{theo:chinese_remainder_phi}).
\end{theo}
\newpage
\noindent
From the above theorem, we relate back to theorem (\ref{theo:square_roots_2}).
\begin{theo}[Chinese Remainder Square Roots]

    Let $n\in\mathbb{Z}^+:2\nmid n$ and $n>1$ be a product of pairwise coprime elements, and $\alpha\in\mathbb{Z}_n^*$, where $\alpha = \beta^2$ for some $\beta\in\mathbb{Z}_n^*$ and $\theta(\beta)=\{\beta_1,...,\beta_k\}$.\\
    
    \noindent
    Then for some $\gamma\in\mathbb{Z}_n^*$, with $\theta(\gamma)=\{\gamma_1,...,\gamma_k\}$, consider:
    \begin{align*}
        \gamma^2 = \beta^2 &\iff \theta(\gamma^2) = \theta(\beta^2) \\
        &\iff (\gamma_1^2, \dots, \gamma_r^2) = (\beta_1^2, \dots, \beta_r^2) \\
        &\iff (\gamma_1, \dots, \gamma_r) = (\pm \beta_1, \dots, \pm \beta_r).
        \end{align*}
    \noindent
    Therefore $\alpha$ has $2^k$ square roots, $\theta^{-1}(\pm\beta_1,...,\pm\beta_k)$.

\end{theo}

\subsection {Square roots of $-1$ modulo $p$}

\begin{theo}[Quadratic Residue $-1$ test (mod $4$)]

    Let $p$ be an odd prime. Then, 
    \Large
    \[\left(\frac{-1}{p}\right)=1\Longleftrightarrow p\equiv 1\pmod{4}\]
    \normalsize
    \noindent
    I.e., $-1$ is a quadratic residue modulo $p$ if and only if $p$ is congruent to $1$ modulo $4$.
\end{theo}
\begin{Note}
    \textbf{Note:} The above refers to the Legendre symbol $\left(\frac{\alpha}{p}\right)$ (\ref{def:le}).
\end{Note}
\begin{Proof}[Quadratic Residue $-1$ test (mod $4$)]
    
    By Euler's criterion $\left(\frac{-1}{p}\right)=1\Longleftrightarrow(-1)^{(p-1)/2}\equiv 1\pmod*{p}$.
    
    \begin{itemize}
        \item If $p\equiv 1\pmod{4}$. Then, $p=4k+1$ for some $k\in\mathbb{Z}$. Taking $1$ from both sides and dividing $2$ yields, $(p-1)/2=2k$.
        Therefore, $(p-1)/2$ is even, thus $(-1)^{(p-1)/2}=1$, as -1 raised to an even power is $1$.
        \item If $p\equiv 3\pmod*{4}$, then $p=4k+3$ for some $k\in\mathbb{Z}$. Then $(p-1)/2=2k+1$ is odd, and $(-1)^{(p-1)/2}=-1$.

    \end{itemize}
     
\end{Proof}

    