\section{The Chinese Remainder Theorem}

\begin{Note}
    \textbf{Note:} $\mathbb{Z}^+$ denotes the set of positive integers, and $\{x_i\}_{i=1}^k$ is short for $\{x_1,...,x_k\}$.
\end{Note}
\begin{theo}[Chinese Remainder Theorem (CRT)]

    Let $\{n_i\}_{i=1}^k\in\mathbb{Z}^+$ all be coprime to each other and let $\{a_i\}_{i=1}^k$ be arbitrary integers. Then there is a solution $a\in\mathbb{Z}$ to the system of congruences:
    \begin{align*}
        a &\equiv a_1 \pmod{n_1} \\
        a &\equiv a_2 \pmod{n_2} \\
        &\hspace*{.5em}\vdots \\
        a &\equiv a_k \pmod{n_k}
    \end{align*}
    
    
    \noindent
    Moreover, if $a$ and $b$ are solutions to the system, then $a\equiv b\pmod{\prod_{i=1}^k n_i}$.
\end{theo}
\begin{Proof}[Solving the System of Congruences (Part 1)]
Let $\{n_i\}_{i=1}^k\in\mathbb{Z}^+$ all be pairwise coprime, and let $\{a_i\}_{i=1}^k$ be arbitrary integers,\\

\noindent
\textbf{Existence:} (i) Construct a partial solution for each congruence. (ii) Each
partial solution must not interfere with other congruences. (iii) Combine partial solutions:\\

\noindent
We by define indexes $i,j=1,...,k$ representing any two $e_1,...,e_k$ integers such that:
 \[e_j \equiv
 \begin{cases} 
 1 \pmod{n_i} & \text{if } j = i, \text{ (target congruence)}\\
 0 \pmod{n_i} & \text{if } j \neq i \text{ (non-interfering)}.
 \end{cases}
 \]
 I.e., $e_j$ has multiplicative identity to it's own system, and additive identity to all other systems
 by being some multiple. This allows us to construct:
 \begin{align*}
    e_1\cdot a_1 &\equiv 1\cdot a_1 \pmod{n_1} \\
    e_2\cdot a_2 &\equiv 1\cdot a_2 \pmod{n_2} \\
    &\hspace*{.5em}\vdots \\
    e_k\cdot a_k &\equiv 1\cdot a_k \pmod{n_k}
\end{align*}
Using additive identity, we concatenate partial-solutions to $a = \sum_{i=1}^k e_ia_i$, the solution.
\end{Proof}

\newpage

\begin{Proof}[Solving the System of Congruences (Part 2)]

\noindent
To construct $e_1,...,e_k$, let $n := \prod_{i=1}^k n_i$ and $n_i^* := n/n_i$. Then, $n_i$ and $n_i'$ are coprime.\\
\end{Proof}