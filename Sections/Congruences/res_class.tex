\section{Residue Classes}
We've spoken before about residue classes in (\ref{def:residue_class}), 
but we'll go into more detail here.

\begin{theo}[Residue Intervals]
   
    Remainders modular $n\in\mathbb{Z}:n>1$, denoted $\mathbb{Z}_n$, is the interval $[0,(n-1)]$.
    As we pass $n-1$, we loop back to $0$. Yielding a general interval of $[x,x+(n-1)]$ for $x\in\mathbb{Z}$.\\

    \noindent
    Adding and multiplying residues shifts to some other position in the interval.
    \begin{itemize}
        \item \textbf{Addition:} $[a] + [b] = [a+b] = [c]\Longleftrightarrow a+b\equiv c\pmod{n}$
        \item \textbf{Multiplication:} $[a]\cdot[b] = [a\cdot b]=[c]\Longleftrightarrow a\cdot b\equiv c \pmod{n}$
    \end{itemize}

    \noindent
    If $n$ is odd, then our interval is $[-n/2,n/2]$. If even, then $[-n/2,n/2-1]$.
\end{theo}
\newpage
\begin{Tip}
    We define \textbf{parity} as the property of an integer being even or odd.    
\end{Tip}

\noindent
\textbf{Example:} Consider tables $\mathbb{Z}_5$ and $\mathbb{Z}_6$:
\begin{table}[h!]
    \setlength{\tabcolsep}{10pt} % Adjust column separation
    \renewcommand{\arraystretch}{1.2} % Adjust row separation
    \centering
\begin{tabular}{|*{14}{c|}}
    
    
    \hline
   \cellcolor{white}$a$ & $0$ & $1$ & $2$ & \cellcolor{OliveGreen!40}$3$ &\cellcolor{OliveGreen!40} $4$ &\cellcolor{OliveGreen!40} $5$ &\cellcolor{OliveGreen!40} $6$ &\cellcolor{OliveGreen!40} $7$ & $8$ & $9$ & $10$ & $11$ & $12$ \\
    \hline
       & $0$ & $1$ & $2$ & $3$ & $4$ & \cellcolor{OliveGreen!20}$0$ &\cellcolor{OliveGreen!20} $1$ &\cellcolor{OliveGreen!20} $2$ & $3$ & $4$ & $0$ & $1$ & $2$ \\
    \cline{2-14}
    \multirow{-2}{*}{$a\mod 5$}&0&-4&-3&\cellcolor{OliveGreen!20}-2&\cellcolor{OliveGreen!20}-1&\cellcolor{OliveGreen!20}0&-4&-3&-2&-1&0&-4&-3\\
    \hline
\end{tabular}
\end{table}

\noindent
Since $5$ is odd, our interval is $[-5/2,5/2]=[-2,2]$, which could be seen as the interval $a\in[3,7]$.

\noindent

\begin{table}[h!]
    \setlength{\tabcolsep}{10pt} % Adjust column separation
    \renewcommand{\arraystretch}{1.2} % Adjust row separation
    \centering
\begin{tabular}{|*{14}{c|}}
    \hline
    \cellcolor{white}$a$ & $0$ & $1$ & $2$ &\cellcolor{OliveGreen!40} $3$ &\cellcolor{OliveGreen!40} $4$ &\cellcolor{OliveGreen!40} $5$ &\cellcolor{OliveGreen!40} $6$ &\cellcolor{OliveGreen!40} $7$ &\cellcolor{OliveGreen!40} $8$ & $9$ & $10$ & $11$ & $12$ \\
    \hline
       & $0$ & $1$ & $2$ & $3$ & $4$ & $5$ &\cellcolor{OliveGreen!20} $0$ &\cellcolor{OliveGreen!20} $1$ &\cellcolor{OliveGreen!20} $2$ & $3$ & $4$ & $5$ & $0$ \\
    \cline{2-14}
    \multirow{-2}{*}{$a\mod 6$}&0&-5&-4&\cellcolor{OliveGreen!20}-3&\cellcolor{OliveGreen!20}-2&\cellcolor{OliveGreen!20}-1&\cellcolor{OliveGreen!20}0&-5&-4&-3&-2&-1&0\\
    \hline
\end{tabular}
\end{table}

\noindent
Since $6$ is even, our interval is $[-6/2,6/2-1]=[-3,2]$, which could be seen as the interval $a\in[3,8]$.\\

\noindent
This interval is no different than $[0,5]$ or $[0,6]$, this shifting of the interval captures $[x,x+(n-1)]$.